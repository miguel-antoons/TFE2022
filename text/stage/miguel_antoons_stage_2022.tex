\documentclass[11pt]{article}
\usepackage[utf8]{inputenc}
\usepackage{lingmacros}
\usepackage{tree-dvips}

\title{Monitoring des données BRAMS et détection automatique des échos de météore}
\author{Miguel Antoons}

\begin{document}

\begin{titlepage}
    \begin{center}
        \LARGE
        \textbf{EPHEC}

        \vspace{0.1cm}
        \LARGE
        Technologie de l'Informatique

        \large
        \vspace{1.5cm}
        \textsc{Avenue du Ciseau 15}\\
        \textsc{1348 Ottignies-Louvain-la-Neuve}

        \vspace*{\stretch{1.0}}

        \line(1,0){350}\\
        \Huge
        \textbf{Rapport de Stage}\\
        \line(1,0){350}\\

        \vspace{0.5cm}
        \Large
        \textit{Miguel Antoons}

        \vspace{0.5cm}
        \large
        2021-2022

        \vspace{2.5cm}
        \large
        \textbf{Maitres de Stage}
        \vspace{0.2cm}\\
        Monsieur Hervé \textsc{Lamy}\\
        Monsieur Antoine \textsc{Calegaro}

        \vspace{2.5cm}
        \large
        \textbf{Professeur Rapporteur}
        \vspace{0.2cm}\\
        Monsieur Arnaud \textsc{Dewulf}


        \vspace*{\stretch{2.0}}
    \end{center}
\end{titlepage}

\tableofcontents

\newpage

\section{Remerciements}
Je tiens à remercier toutes les personnes qui m'ont aidé tout au long du stage et lors de la rédaction de ce rapport.\\
\\
En premier lieu, je remercie Mrs. Hervé Lamy et Antoine Calegaro. En tant que maîtres de stage, ils m'ont beaucoup appris et ont partagés leurs connaissances dans le domaine de l'informatique.\\
\\
Je remercie également mon professeur rapporteur, Mr. Arnaud Dewulf qui m'a permis de trouver ce stage et qui m'a suivi tout au long de son déroulement.\\
\\
Enfin, je voudrais exprimer ma reconnaissance envers toutes les personnes qui m'ont conseillé sur, et ont relu mon rapport de stage.

\newpage

\section{Introduction}

\subsection{Objectifs du Stage}
Le stage d'insertion professionnelle proposé par l'Ephec a pour objectif de familiariser les étudiants avec le milieu dans lequel ils valoriseront leur diplôme.
Il permet de mettre en pratique la matière théorique vue tout au long du cursus en technologie de l'informatique grâce à la confrontation à des situations concrètes, mais également à l'aide de l'approche d'un professionnel.

\subsection{Attentes Personnelles}
% ! don't forget to mention what projet BRAMS is
Cette offre de stage a été proposé lors d'une présentation organisée par les membres du projet BRAMS en question durant le cours de traitement de signal donné par Mr Dewulf.
Ayant déjà contacté des sociétés, mais n'ayant pas encore pris de décision, j'ai décidé de prendre rendez-vous avec la société.\\
\\
Le travail principal proposé par la société était la migration d'un site web vers un autre framework.
L'ancien framework ainsi que le nouveau m'étant inconnu, ceci m'a attiré pour les nombreuses compétences à apprendre.\\
Ce stage, comme la plupart des stages, permettait également de voir comment un département informatique s'organise et fonctionne.\\
\\
Enfin, n'étant plus dans un milieu d'entrainement, j'étais curieux de la complexité d'écrire des programmes complets, de qualité dans les délais imposés.
Ceci s'étend à partir d'un débogage et une maintenance facilités pour le développeur qui viendra après moi, jusqu'au développement d'une interface conviviale et facile à utiliser pour l'utilisateur.
Cette attente était d'autant plus intéressante sachant que j'allais devoir apprendre un nouveau langage et un nouveau framework durant ces délais.


\end{document}